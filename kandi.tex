% Latex-pohja kandidaatintyölle
% Alkuperäinen pohja Joona Widgren, muokkaukset Juho Heinisuo
%
% MIT -lisenssi

\documentclass[12pt,a4paper,finnish]{article}

% Makropakettiesimerkkejä
\usepackage{setspace}
\onehalfspacing{}
\usepackage{amsthm}
\usepackage{amsfonts}         
\usepackage{amsmath}
\usepackage{amssymb}
\usepackage[top=3cm, bottom=3cm, left=4.2cm, right=4.2cm]{geometry}
\usepackage{graphics}
\usepackage{hyperref}
\usepackage{listings}
\usepackage{multicol}
\usepackage{verbatim}
\usepackage{pdfpages}
\usepackage{babel}
\usepackage[utf8]{inputenc}
\usepackage[T1]{fontenc}

\newcommand{\R}{\mathbb{R}}
\newcommand{\C}{\mathbb{C}}
\newcommand{\Q}{\mathbb{Q}}
\newcommand{\N}{\mathbb{N}}
\newcommand{\No}{\mathbb{N}_0}
\newcommand{\Z}{\mathbb{Z}}
\newcommand{\diam}{\operatorname{diam}}

\theoremstyle{plain}

\theoremstyle{definition}
\newtheorem{maar}{Määritelmä}
\newtheorem{esim}[equation]{Esimerkki}

\theoremstyle{remark}
\newtheorem{huom}[equation]{Huomautus}

\theoremstyle{figure}
\newtheorem{kuvio}[equation]{Kuvio}

% Tästä alkaa itse dokumentti
\begin{document}

% Kansilehti
\begin{titlepage}
\linespread{1}
\title{\vspace{20px} \hspace{0px}Otsikko \hspace{0px}}
\date{}
\maketitle
\thispagestyle{empty}
\vspace*{350px}
\begin{tabbing}
\hspace*{300px} \= \\
\> \underline{etunimi} toinen-nimi sukunimi \\ % Kutsumanimi alleviivattu
\> opiskelijanumero  \\
\> Helsingin yliopisto \\
\> Valtiotieteellinen tiedekunta \\
\> Taloustieteen oppiaine \\
\> Palautusaika % Kuukausi vuosi
\end{tabbing}
\end{titlepage}
\pagebreak

% Sisällysluettelo
% Generoituu automaattisesti otsikoiden perusteella
\tableofcontents
\pagebreak

\section{Johdanto}

Tekstiä

\pagebreak

\section{Otsikko}\label{}

Tekstiä

\subsection{Alaotsikko}

Listaesimerkki

\begin{enumerate}

    \item Listaelementti

    \item Listaelementti, jossa footnote\footnote{Footnote}

    \item Listaelementti, jossa inline-matematiikkaa $x_i = y_i * \pi$

\end{enumerate}

\subsection{Alaotsikko}

Taulukkoesimerkki

\begin{tabular}{l l}

\textsc{otsikko 1} & \textsc{otsikko 2} \\

    a & x \\
    b & y \\
    c & z \\

\end{tabular}

Taulukkoesimerkki matematiikan kanssa

\begin{tabular}{l l l}
    \textsc{merkintä} & \textsc{kaava} & \textsc{selitys} \\

    $(F/P, i, n)$ & ${(1+i)}^n$ & Nykyarvosta tulevaisuuden arvoon \\
    $(P/F, i, n)$ & ${(1+i)}^{-n}$ & Tulevaisuuden arvosta nykyarvoon \\
    $(F/A, i, n)$ & $\frac{{(1+i)}^n-1}{i}$ & Yhden periodin arvosta tulevaisuuden arvoon \\
    $(A/F, i, n)$ & $\frac{i}{{(1+i)}^n-1}$ & Tulevaisuuden arvosta yhden periodin arvoon \\
    $(A/P, i, n)$ & $\frac{i{(1+i)}^n}{{(1+i)}^n-1}$ & Yhden periodin arvoon nykyarvosta \\
    $(P/A, i, n)$ & $\frac{{(1+i)}^n-1}{i{(1+i)}^n}$ & Nykyarvoon yhden perioidin arvosta \\
    
\end{tabular}

\subsection{Alaotsikko}

\subsubsection{Alaotsikon alaotsikko}

Matematiikkaesimerkki

\begin{equation}
    L = q(A/P, i_l/12, 12l)
\end{equation}
 
Toinen matematiikkaesimerkki
 
\begin{equation}
    C(N) = \sum_{t=0}^{N-1} K(P/A, i_f/12, 12)(P/F, i_f, t)
\begin{equation}
    PV_b(N) = s_b + C_b(N) + V(N) - P(N)
\end{equation}

\section{Otsikko}\label{}

Tekstiä

Kuvan voi liittää tiedostoon esimerkiksi näin (katso .tex-tiedoston kommentit):

\centerline{\includegraphics[width=0.8\textwidth]{path/to/image}}
\begin{kuvio}
    Kuvan selitys
\end{kuvio}

\section{Johtopäätökset}\label{concl}

Tekstiä

\pagebreak

\section*{Lähteet}
\addcontentsline{toc}{section}{\hspace{18px}Lähteet}

Glaeser, Edward L.; Gyourko, Joseph; Saks, Raven E. (2005).\\
\textit{Why Have Housing Prices Gone Up?}\\
American Economic Review. Volume 95 – Number 2.\\  

\parindent=0pt

Sinai, Todd; Souleles, Nicholas S. (2005).\\
\textit{Owner-occupied Housing as a Hedge against Rent Risk.}\\
Quarterly Journal of Economics. Volume 120 – Number 2.\\  

\parindent=0pt

jne.

\pagebreak

\textsc{Liitteet}

Liitteen tekstit

\end{document}

